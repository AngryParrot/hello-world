%!Tex root = ./Main.tex
\subsection{Can Norms Conflict?}

Consider that a teacher is blind and needs a guidance dog \citep{mclauhghlin_2016}. For that reason we grant her the right to be accompanied by a guidance dog. Consider also, that some people have severe allergies to dogs and enjoy on that ground a right not to be burdened by the presence of a dog. One of the teacher's pupils has a severe allergy towards the guidance dog. Just being in the same room causes heavy breathing, watering eyes, and continuous sneezing. Which right is to be respected here? And, even more fundamental, can such a situation even occur?

According to specificationist it cannot. Rights are absolute and are not weighted against each other. The specificationist also holds that rights are only genuine within a certain domain. A specificationist would hold that there is no general right to be killed. Put differently, the right not to be killed is not always valid. For example, if you were to attack somebody with a knife, then you do not enjoy the said right. The specificationist would specify the context in which you enjoy a right: ``You have a right not to be killed, unless you were to attack somebody with a knife.''

But the specificatist's view is challenged. We only mention one objection here, as it has direct importance to our endeavour. Since specificationists hold that norms do not conflict, they must list all the different contexts in which one enjoys a right or is under a duty and when one does not. Feinberg \citet[p.~221-251]{feinberg2014rights} and \citet[p.~82-104]{thomson1990realm} argue that fully specified norms are not knowable, because we would have to consider every possible situation that might arise and has import on the validity of the norms, ex ante. 

In practice, a specificationist puts forward a norm. Then someone tries to find a context which the norm yields a false result. Either it would apply in the context, when it should not, or it does not apply, when it should. Then the norm becomes more specified as it is supplemented with the new context. This procedure is akin to Rawls' reflective equilibrium \citep{rawls1999justice,mcdermott2008analytic}. 

The problem for the computer scientist is, however, that she needs to design a machine \emph{right now} and she cannot wait until philosophers come up with the fully specified right (if this is even feasible). That means that in practice we will most likely encounter situations in which norms conflict. In section \ref{sec:dls} we formalise these situation as the problem of Deontic Lock States and provide an option how to deal the problem. However, our framework can also be used by specificationists. For them the problem of DLS does not occur, as they have fully specified their norms. 

%Regardless which position one takes. Our framework is capable of providing an adequate representation and solution